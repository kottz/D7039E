
\subsection{Robotic arm}
In the beginning many different arm choices were discussed. One solution was using a simpler forklift design that just moves up and forward in a rigid motion without rotation.
This solution would suffice. However, the decision was made to go with a arm design with links connected with motors.
This is a more complex solution but opens up for translation and rotation in 3D-space.
The design allows the arm to move to any spot in the defined work space.
to pick up the piece at an angle from factory.
A limitation is the fact that the last arm joint is kept static and parallel to the ground, So the end effector cannot rotate. 
This can be adjusted by adding one more motor. However, for the task at hand it deemed not necessary.

\subsection{End effector}
In order to pick up the object a gripping mechanism was needed, different choices were discussed in the early stages.
However the simple solution using a claw would suffice.
By rotating the shaft of the motor, connected to the claw, the claw either retracts or expands.
As stated above this design is easy to implement and works well.
One downfall is the fact that the connecting rods that moves the claw are fragile and can break.
This can be improved by using metal instead of plastic for the rods connected between motor and claw. 


