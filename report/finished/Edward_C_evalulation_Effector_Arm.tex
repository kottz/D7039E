
\subsection{Robotic arm}
In the beginning many different arm choices was discussed. One solution was using a simpler forklift design that just moves up and forward in a rigid motion without rotation.
This solution would suffice. However, we decided to go with a arm design with links connected with motors.
This is a more complex solution but opens up translation and rotation in 3D-space.
The design allows the arm to move to any spot in the defined work space.
to pick up the piece at an angle from factory.
A limitation is the fact that we decided to keep the of the arm static and parallel to the ground.
So we can't rotate the end effector. By adding one more motor this can be done. However, we deemed it not necessary for the task at hand.

\subsection{End effector}
In order to pick up the object a gripping mechanism was needed, different choices was discussed in the early stages.
However, we settled for the simple solution of using a claw.
By rotating the shaft of motor the claw either retracts or expands.
As stated above this design is easy to implement and works well.
One downfall is the fact that the connecting rods that moves the claw are fragile and can break.
This can be improved by using metal instead of plastic. 


