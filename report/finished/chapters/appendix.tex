% !TeX root = ../main.tex

\section*{Martin Blaszczyk}
Martin is a 5th year Y-student with interest in Control och Mechatronics. 
In this course he'll take the role of the Project Leader where the main objectives
are to keep focus on the goal, hold meetings and an overall oversight of the project. 
As for the technical part the main interest will be in machine vision together with
Edward K. to use cameras or other sensors to localize external objects for the 
robot to grip, avoid or approach.

\section*{Edward Cedgård}

Master in  electronic systems and control engineering.
Edward's main task is to design the robotic arm and gripper mechanism together with Niklas. 
Tasks as deriving the kinematic equations, and implementation using forward and inverse kinematics.
Choise of motors, armdesign, communication with motors using serial communication. 

\section*{Niklas Dahlquist}
Niklas is studying his fifth year at the Engineering physics and electrical engineering student program.

His main focus will be to work with Edward Cedgård to evaluate the gripping mechanism and if necessary design new components and model the corresponding control system to be able to lift up and hold the target object.

\section*{Edward Källstedt}
Currently studying his fourth year in the Master Programme in Computer Science and Engineering.
A fan of making things secure, fast, scalable, and well-documented. Primarily interested in
low level software development. Will initially work on the machine vision implementation 
together with Martin. In addition to machine vision specifically this work will also consist
of robot localization and collision avoidance. As the project progresses he will take on more
general software problems that might arise. The first week will be spent researching different
computer vision technologies.


\section*{Albin Martinsson}
Albin is a 5th year computer science student specializing in industrial computersystem. 
In this project he will be focusing on the arrowhead integration and bein charge of the Github repository.  
This will entail connecting all the services toeach other and making sure they are authenticated and secure.  
Being in chargeof the git repository will entail merging pull requests and sorting out conflicts,
making sure that the version control part of this project runs smoothly.


\section*{Måns Norell}
Studying for a master in electronic systems and control engineering.
Måns main task is to design the base and line-following controller for moving the robot along a line. 
Tasks include designing the base, printing the specialized parts, simulating and testing the base.
Communication between controller, motor and camera will be worked on in collaboration with those in charge of these tasks. 


\section{Working together}
\section*{Project structure}
To keep the project going and have an organized structure the project is divided 
in different parts, or subprojects. Each group member is either alone or in group responsible for each part of the project which coincides with their interests. 
\begin{itemize}
    \item Arrowhead
    \item Machine vision and localization of external objects
    \item Gripping tool
    \item Movable base
\end{itemize}

\section*{Meetings}

Every week the group met on Mondays and Thursdays to catch up and support each other. This structure gave the students a great deal of responsibility to do work for each meeting while still maintaining a good structure of the project and encouraged discussions. 
The agenda of the monday meeting was:
\begin{itemize}
    \item Status of work done the previous week by each member
    \item Preparation for the seminar
    \begin{itemize}
        \item Discussion of the previous seminar meeting
        \item How the weeks work has been coinciding with the seminar feedback
        \item Questions to ask the teachers
        \item Questions to ask the other group
        \item Who does what during the seminar
    \end{itemize}
    \item Other
\end{itemize}
The Thursday meeting was to collect and reflect over the 
feedback from the teachers and the other group from the seminar. Also a status on the work planned to be done the following week was discussed so that each member had a good understanding of what the other members were doing. Those meetings had had the following agenda
\begin{itemize}
    \item What feedback did the teachers give
    \item What feedback did he other group give 
    \item Feedback to each other withing the group
    \item Work to be done the following week
    \item Other 
\end{itemize}

\subsection{Project planning}
The projects course started in August 2020 and continued until mid January 2020 with the project deadline in the middle of December. To keep a good structure and to synchronize all the parts of the project a plan was made, shown in Table \ref{tab:overall_time_plan}. This plan enabled the group to be flexible and maintain a good long-term structure of the project. Every month a more detailed time plan was prepared to facilitate the small changes, delays etc. 
\begin{frame}
    \subsection{Time plan}
    \frametitle{Overall timetable}
    \begin{table}
        \begin{tabular}{| l | c | c | c | c }
            
            Sep & Oct & Nov & Dec \\
            \hline \hline
            Concept generation & Evaluation & Evaluation &  \\ 
            \hline
            Theory & Prototyping & Evaluation & Finishing up \\
            \hline
            Simulation & Evaluation & Evaluation & \\
            \hline
            Prototyping & Final Design & Evaluation &  \\
            \hline
 
        \end{tabular}
    \end{table}    
\end{frame}

\begin{frame}
    
    \frametitle{Time plan for October}
    \begin{table}
        \begin{tabular}{l | c | c | c | c }
        Subproject & Week 1 & Week 2 & Week 3 & Week 4 \\
        \hline \hline
            Arrow & Working & Structure & Fusion & ...\\
            Base & CAD & Communication & ROS & Final\\
            Arm  & Assembly & Gripper & ROS & Final\\
            MV & Porting to NVIDIA & API & Evaluation & ...\\
        \end{tabular}
    \end{table}
\end{frame}



For the group members to what task were to be done, the built in function of Issues on the souce control platform Github\textregistered \ was utilized. A Milestone was created for each week and populated with issues. When an issue was finished it could simply be closed. If the issue was delayed it showed clearly in the Issue overview which tasks had to be prioritized. 

\subsection{Source control}
To keep track of the different software implementation the projects source control implements Git in one common repository \cite{repo}. The repository is where all source code and relevand 3D-files are located. This report is written in LaTeX with Git as source control. To make sure it's easy for all members to write their designated sections a workflow was designed to to minimiza merge conflicts while writing drafts as show in Fig. \ref{fig:git_workflow}. Without this flow the group members would experience merge conflicts after every push which would make it more complex and time consuming.

    \begin{figure}
        \begin{center}
\resizebox{6.0cm}{!}{

\begin{tikzpicture}
    [align=center, auto]
    \node [computing] (start) {\textbf{CHAPTER\_descriptive\_filename.tex}};
    \node [computing, below= of start] (push) {Push to \textbf{/sections}};
    \node [computing, below= of push] (finished) {Finished?};
    \node [computing, below= of finished] (write) {Write on section}; 
    \node [computing, below= of write] (commit) {Commit};

    \node [computing, right= of finished, xshift=5em] (branch) {New branch \\ \textbf{REVIEW\_chapter}};
    \node [computing, below= of branch] (paste) {Paste section in \\ \textbf{/chapters/chapter.tex}};
    \node [computing, below= of paste] (pushb) {Push new branch};
    \node [computing, below=of pushb] (PR) {PR against \textbf{report-unsafe}};
    \node [computing, below= of PR] (review) {Wait for review};
    
    



    \coordinate [left= of push] (lrpush);
    \coordinate [left= of write] (lrwrite);
    \coordinate [left= of commit] (lrcommit);
    \coordinate [left= of finished] (lrfinished);
    \coordinate [right= of PR] (crPR);
    \coordinate [below= of crPR] (crreview);
    



    \draw [->] (commit) -- (lrcommit) -| (lrpush) -- (push);

    \draw [->] (start) -- (push);

    \draw [->] (push) -- (finished) -- node[midway, fill=white] {No} (write) -- (commit);

    \draw [->] (finished) -- node[midway, fill=white] {Yes} (branch);

    \draw [->] (branch) -- (paste) -- (pushb) -- (PR)  -- (review);

    \draw [->] (review) -| (crreview) -- node[midway, xshift=2.2em, yshift=-0.5em, fill=white] {Fix} (crPR) -- (PR);
    
    
\end{tikzpicture}
}
\end{center}
\caption{Report writing flowchart}
\label{fig:git_workflow}
\end{figure}











