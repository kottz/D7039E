\section*{Prototype components}
Overview of what is being used in the evaluation
\section*{Movable base evauation}
\section*{Arm evaluation}
\section*{Gripping evaluation}
\section*{Sensor evaluation}
\section*{Machine vision evaluation}
The line following algorithm has generally worked well.
During testing the robot has been able to follow the lines and identify the QR codes.
The robot is able to travel between points in the grid in a sufficient manner.

%Limitation 1
A limitation of the line following algorithm is its sensitivity to noise.
Bad lighting conditions and a poorly calibrated color mask can negatively affect the ability to track the line reliably.
A possible future optimization to mitigate this effect would be to keep track of the center point of the line in the previous frame.
The image moment calculation could then be done in a smaller subsection of the horizontal slice around the previous center.
Currently, the algorithm uses the entire width of the frame to calculate the center point.
This leads to a heightened sensitivity to random noise in the outer parts of the frame.
If the general position of the line is already known only pixels close to this point should be considered, since the line will not move considerably between frames.

% Limitation 2
Also, another limitation is the fact that the machine vision system is dependent on sufficiently fast hardware.
Some lag will always be introduced while the current image is being processed.
If the hardware on which the system is run is slow this lag runs the risk of being alarmingly large.
This will result in the angle sent to the base controller being out of date.
However, when running on a Raspberry Pi 4 or the Nvidia Jetson Nano this delay is negligible.
This can also be tuned by changing the image resolution used to capture frames with the camera.
Preferably, the resolution should be low to increase performance but not so low that the ability to recognize QR codes is impaired.
During testing $360$ by $240$ was found to be a good middle ground between performance and image quality.
\section*{Arrowhead evaluation}