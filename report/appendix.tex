\section{Team members}

\subsection{Martin Blaszczyk}
Martin is a 5th year Y-student with interest in Control och Mechatronics. 
In this course he'll take the role of a Project Leader where the main objectives
are to keep focus on the goal, hold meetings and an overall oversight of the project. 
As for the technical part the main interest will be in machine vision together with
Edward K. to use cameras or other sensors to localize external objects for the 
robot to grip, avoid or approach. 

\subsection{Edward Cedgård}
Master in engineering electronic systems and control engineering.
Edward main task is to design the grip mechanism with Niklas. Topics like
the design of the claw, nessesary components, model of the controlsystem and then implementation. 

\subsection{Niklas Dahlquist}
Niklas is studying his fifth year at the Engineering physics and electrical engineering student programe.

His main focus will be to work with Edward Cedgård to evaluate the gripping mechanism and if necessary design new components and model the corresponding control system to be able to lift up and hold the target object.

\subsection{Edward Källstedt}

\subsection{Albin Martinsson}
Albin is a 5th year computer science student specializing in industrial computer system. In this project he will be focusing on the 
arrowhead integration and be in charge of the github repository. This will entail connecting all the services to each other and
making sure they are authenticated and secure. Being in charge of the git repository will entail merging pull requests and sorting
out conflicts, making sure that the version control part of this project runs smoothly.

\subsection{Måns Norell}
As Måns is a control student, he will be working on the movement and navigation system for the robot. In the beginning this will manifest in moving the robot to user specified coordinates.

\section{Working together}
To keep the project going and have an organized structure the project is divided 
in different parts. 
\begin{itemize}
    \item Arrowhead
    \item Machine vision ans localization of external objects
    \item Gripping tool
    \item Movable base
\end{itemize}
Each group member is either alone or in group responsible for each part of the 
project which coincides with their interests. 
Every week the group will be meeting on Mondays and Tuesdays. 
The monday meetings will have the following agenda 
\begin{itemize}
    \item Status of work done the previous week by each member
    \item Preperation for the seminar
    \begin{itemize}
        \item Discussion of the previous seminar meeting
        \item How well the weeks work has been coinciding with the seminar feedback
        \item Questions to ask the teachers
        \item Questions to ask the other group
        \item Who does what during the Tuesday seminar
    \end{itemize}
    \item Other questions
\end{itemize}
The tuesday meeting will be after the seminar to collect and reflect over the 
feedback from the teachers and the other group. Also a status on the work planned to be done 
the coming week will be discussed so that each member has an overview of what 
the other members are doing. The meetings will have the following agenda
\begin{itemize}
    \item What feedback did the teachers give
    \item What feedback did he other group give 
    \item Feedback to eachother withing the group
    \item Work to be done the following week
    \item Other questions
\end{itemize}









