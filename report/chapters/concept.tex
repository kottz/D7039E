\section*{Mechanical structure}
The final design of the robot. Include CAD renderings etc. 
\section*{Electrical components}
Electrical components such as motors, MCUs etc. Why were they chosen

% Putting this here temporarily. Should probably be moved in the future.
\section*{Requirements}
\subsection*{Computer Vision System}
\textbf{ID:} 01 \\
\textbf{Titel:} \emph{Line Detection}
\begin{itemize}
    \item Given a distinctly colored line along the floor the computer vision system should be able to track it and
    feed back an accurate measurement of how much the robot deviates from said line.
\end{itemize}
\textbf{ID:} 02 \\
\textbf{Titel:} \emph{Color Detection}
\begin{itemize}
    \item Camera input should be filterable by RGB pixel values. Objects within certain color ranges should be detectable.
\end{itemize}
\textbf{ID:} 03 \\
\textbf{Titel:} \emph{QR Code Identification}
\begin{itemize}
    \item The system should be able to recognize and read the contents of QR Codes.
\end{itemize}
\textbf{ID:} 04 \\
\textbf{Titel:} \emph{Real-Time Performance}
\begin{itemize}
    \item It should be possible for the system to keep up with and process a continuous video stream in real-time.
\end{itemize}
\textbf{ID:} 05 \\
\textbf{Titel:} \emph{Visualization}
\begin{itemize}
    \item An accompanying GUI should exist where the raw video stream can be seen adjacent to a video stream where color
    and line detection is active. Detected objects should have a border and an object coordinate should be seen on screen.
    When the system is following a line it should be possible to see how the robot is positioned relative to the line.
    A measurement of the current deviation should be seen on the screen.
\end{itemize}
