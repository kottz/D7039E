When workign with computers or cameras one has probably used the RGB-color space. The acronym RGB refers to the three colors Red, Green and Blue respectively and it's how camera sensors detect the incoming light. While the RGB-space is easy to understand it doesn't really represent how humans\dots

The most popular and well known color space is probably the RGB-color space (Red, green and Blue) as it's how the cameras detect and record the incoming light. While the RGB-space is easy to understand it doesn't reflect well how humans perceive colors. The term \textit{chromaticity} descibes the color quality, regardless it's luminance. The way human perceive chromativity seems to be mainly by considering the two characteristics, \textit{hue} and \textit{saturation}. Hue refers to the dominant color and saturation refers to the absense of white light, this led to the development of HSV-space (Hue, Saturation, Value) where Value can be seen as the brightness. This color space is much better suited to tune the machine vision algorithm. And example is that if we'd like to increase the red color in RGB space we'd have decrease the green and blue values, not the red. While in the HSV-space we'd only increase the saturation to make the algorith to drtect more red. 

