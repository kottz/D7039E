The most popular and well known color space is probably the RGB-color space (Red, green and Blue) as it's how the cameras detect and record the incoming light. While the RGB-space is easy to understand it doesn't reflect well how humans perceive colors. The term \textit{chromaticity} descibes the color quality, regardless it's luminance and the way human perceive chromativity seems to be mainly by considering the two characteristics, \textit{hue} and \textit{saturation}. Hue refers to the dominant color and saturation refers to the absense of white light. This led to the development of HSV-space (Hue, Saturation, Value) where Value can be seen as the brightness. This color space is much better suited to tune the machine vision algorithm to find the changes of color. 
