preliminary stuff
\begin{itemize}
    \item Since the base use tracks, it can turn on a point. Therefor the turning radius will be the length from the center of the tracks, to the object furthest away from the center.
    \item The acceleration need to be slow enough to not tip the robot over, since it seems quite top heavy.
    \item The motors will need decoders for some of the control to work, if done the way they're planed
    \item The motors also need to be strong enough to move the robot, both with and without load.
\end{itemize}

Since the robot is using tracks, it can be viewed as a simple two motor system. Where one can control the turning
 and direction of the robot by reducing or increasing the speed of the two motors. The diffirence in the speed ($\Delta v$)
 can then be used to calculate the turning speed ($\dot{\theta}$)