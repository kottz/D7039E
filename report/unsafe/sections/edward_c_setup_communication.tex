\section*{communication}
Unlike most servo motors Dynamixel uses digital packet communication instead of pulse-width modulation (PWM) signals.
By grouping data into different packages and transmitting it as a bundle.
This in turn opens up for many possibilities since the motors responds to commands,
have built in control functions together with memory allocation on the motor itself.
Dynamixel recommends two different setups for communication,
either by using a Tri-state-buffer to construct your own communication bridge  based on Universal asynchronous receiver-transmitter setup (UART). Or by using their own solution called U2D2.
In essense, the U2D2 uses the same protocol but also enables for USB communication between the controller and Dynamixels\cite{robotis}.

\subsection{UART full duplex to half duplex}
UART communication scheme works by connecting the transmitter pin (Tx) of the Jetson Nvidia Nano microcontroller (MCU)
directly to the receiver pin (Rx) of the motors and vice versa.
UART communication can be setup into three different ways of communication \cite{duplex}.
Full duplex, half duplex and simplex.
Full duplex allows for data communication both ways simultaneously.
Half duplex only allows for one way communication at any given time, by either sending or receiving.
Simplex communication is static, meaning communication is always one direction.
\newline
Full duplex and half duplex are not directly compatible. 
If the MCU and Dynamixel motors are connected directly,
since the MCU sends and receive data simultaneously it will occupy the transmission line indefinitely and the Dynamixel won't be able to tell when the messages stop. 
This phenomena is known as chatter. 
The solution for this is to implement a tri-state-buffer circuit converting the full system to half duplex.

\subsection{Tri-state-buffer}

A tri-state-buffer dictates when the MCU is allowed to communicate with the motors. This explained  by figure \ref{duplex}.
By controlling the direction pin (GPIO pin),
we can direct the flow of data. 
If the direction pin is pulled high we simultaneously allow for transmission from the MCU (Tx) to the motors and disable the MCU receiver pin (Rx).
A crucial part is timing, the direction pin must be pulled high long enough for the data to transfer through the buffer otherwise information is lost.
A logic analyzer will give this information and is recommended to use in conjugation with the tri-state-buffer for tuning.

\subsection{Data package - structure of message}
The protocol allows for a size of 8 bits per message between the MCU and motors. 
The two types message structures is represented by the figures below \ref{instr} and \ref{return}.
For more information how to interpret the structure use dynamixels own documentation ().



