\subsection{Arrowhead framework}
Arrowhead framework is a framework designed to aid in the functonality of Iot-devices, havning a focus on large scale industry and
productivity. The framework provides a number of core system, systems that are required to run to enable certain functionality. This 
project uses five of them nameley, service registry, orchestrator, authorization, gateway and gatekeeper. The service registry is used 
to register systems and services, a system is defined as either a provider or a consumer. A provider provides a service for examample 
the reading of a temperature sensor and in turn the consumer consumes that service which in this example might be to log the temperature
in a database. A service is therefore what the provider is providing and the consumer consuming, more on this later. The orchestrator is
used to decide which consumer and acces which provider and what services said provider is entitled to provide the conusmer with. The
authorization is used to declare which providers are authozired within both the local cloud, intra cloud authorization, and between
differnt clouds, inter cloud authorization. When defining inter or intra cloud rules to user also has to decide which interface should 
handle the communication between the systems. These three systems are the minimun requirment for running a local arrowhead cloud. 
To enable intra cloud communication both the gateway and gatekeeper core systems will have to be running as well. 

\subsection{Communication}
This project uses the arrowhead framework for Communication between the robot and the local arrowhead cloud as well as the 
remote cloud from the factory. This section of the report aims to explain how the communication between these different clouds works as well
as to define the purpose of using the arrowhead framework. This project decided to use arrowhead framework for communication between 
the local cloud and the robot, all internal communication on the robot will be handled using ROS. This proides faster and more robust 
control of the robot, not relying on internet connection and so forth to control the robot. It should be noted that the arrowhead framework
is fully competent to control every aspect of the robot as well, but can run in to problems with delay in time critical sections.

\subsubsection{Intra cloud communication}
The prerequisites for using intra cloud communication is:
\begin{itemize}
    \item Running core services i.e. service registry, authorization and orchestrator.
    \item A registered consumer and provider.
    \item Defined intra cloud rules.
\end{itemize}

\subsubsection{Inter cloud communication}
The prerequisites for using inter cloud communication is:
\begin{itemize}
    \item Running core services i.e. service registry, authorization, orchestrator, gateway and gatekeeper.
    \item A running ActiveMQ relay server.
    \item Defined inter cloud rules.
\end{itemize}

\subsubsection{Purpose of using arrowhead framework}
The communication between different IoT-devices has been an issue for quite some time, and the security issues surrounding IoT-devices
are well known. Arrowhead framework attempts to provide a soulution for fixing these issues. Using HTPPS with certificates the arrowhead
framework provides a solution for tackling the basic secruity flaws of oth er unprotected IoT-devices, this means that you would have to have
the right certificate and the keyword for that certificate in order to access the service registry. The same applies to both the orchestrator
and authorization core system. 

\subsubsection{Services}
Two services will be offered by the robot, which acts as a provider i.e. it provides a service. In this case the services offered
is place and pick-up. Both these services will have to be registered in the service registry where they will have differnt URI:s 
to and servicedefinitions to tell them apart. Otherwise they will work very much the same. They will, when prompted to do so, provide a
service to either pick-up the piece from the end of the factorys conveyer belt or place it at the beginning of the conveyer belt. 
In both cases the services will wait for a JSON containing if the service should be carried out or not. When recieving a JSON containing that the service will be carried out. Both the services
works in a similar way:

\begin{itemize}
    \item Use pathfinding to go to predefined goal coordinate.
    \item Perform desired action i.e. pick up or place the piece.
    \item Await further instructions.
\end{itemize}

\subsection{Python implementation of the arrowhead framework}
A python implementation of the arrowhead framework core systems was created because:
\begin{itemize}
    \item Current implementation on the arrowhead framework's github had no support for HTTPS.
    \item Gain a further understanding of how the framework actually works.
    \item Facilitate the integration between ROS, i.e. being able to easily run the provider from the robot. 
\end{itemize}

The python implementation consists of four major parts. The first part is a class containg all the methods for registering, authorizing and
orchestrating services. The second part is a JSON-file containing all the properties for both provider and consumer i.e. path to certificate, 
password to certificate, system name, address and port. It also contains service definitions, details for orchestration and intra cloud
authorization. The third part is a skeleton for a provider, letting the user know in which order to execute the different commands. 
That order is: 
\begin{itemize}
    \item Register the provider system in the service registry.
\end{itemize}
The forth and last part is a skeleton for a consumer, letting the user know in which order to execute the different commands.
That order is:
\begin{itemize}
    \item Register the consumer system in the service registry
    \item Register the service definition in the service registry.
    \item Create the intra cloud authorization rules.
    \item Create the orchestration store entry.
    \item Start the orchestration based on the consumer's id.
    \item Recieve the provider port and address.
    \item Recieve the service URI.
\end{itemize}

The class contains a great variety of helper methods to make as easy for the user as possible. Only the most necessary 
and non-dynamic information is stored in the JSON. It uses the names of systems, services, and service definitions to find the necessary
id's, which means that if other services or systems are added or deleted the skeletons will still work. 



